\chapter{Úvod}

Generování softwarových komponent bude využívat generativního programování. Generativní programování je způsob programování, kdy je zdrojový kód programu generován na základě šablony. Tuto šablonu tvoří různé vzory.  

K vytvoření šablony, pro vygenerování finálního zdrojového kódu bude v práci využíván model vlastností. Model vlastností zachycuje všechny různorodosti a podobnosti všech možných variant finálního produktu. Různé charakteristiky, které finální produkty rozlišují se nazývají právě vlastnosti. 

Cílem této práce bude vygenerovat šablonu jako model vlastností ze zápisu gramatiky jazyka tesa pomocí frameworku Xtext a na základě vybraných vlastností z tohoto modelu vygenerovat finální zdrojový kód a ušetřit tak programátorům čas strávený přepisováním zdrojových kódů při změně gramatiky nebo změně konfigurace, či vytvoření nových variant konfigurace.

Čtenář je v úvodní části seznámen s problematikou modelování vlastností, nástroji které jsou pro modelování, nebo tvorbu modelů využívány. Dále bude seznámen s problematikou generativního programování a frameworkem Xtext, který je jednoduchým frameworkem pro tvorbu vlastního jazyka.