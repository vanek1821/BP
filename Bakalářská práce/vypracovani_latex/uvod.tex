\chapter{Úvod}
Tento dokument se vztahuje k předmětu PRJ5, což je předmět související s bakalářskou prací. Tento dokument obsahuje analýzu problému, ve kterém pojednává o modelování vlastností, generativním programování a Xtextu. Ve finální bakalářské práci bude doplněn návrh implementace, popis řešení a zhodnocení dosažených výsledků.

Generování softwarových komponent bude využívat generativního programování. Generativní programování je způsob programování, kdy je zdrojový kód programu generován na základě šablony. Tuto šablonu tvoří různé vzory.  

K vytvoření šablony, pro vygenerování finálního zdrojového kódu bude v práci využíván model vlastností. Model vlastností zachycuje všechny různorodosti a podobnosti všech možných variant finálního produktu. Různé charakteristiky, které finální produkty rozlišují se nazývají právě vlastnosti. Vlastnosti jsou základními stavebními kameny modelu vlastností, který zobrazuje závislosti mezi nimi.  

V úvodní části se tato práce zabývá analýzou problematiky modelů vlastností, nástroji které se pro modelování vlastností využívají, generativního programování a Xtextu. Následně bude popsán návrh implementace, popis samotné implementace a na závěr zhodnocení dosažených výsledků.


