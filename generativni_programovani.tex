\chapter{Generativní programování}
Generativní programování je druh programování, který odděluje model od implementace. Realizuje se pomocí generátorů. Generátor využívá modelu jako šablony, ze které je generován výsledný kód. Generátory jsou založeny na modelech, které definují sémantiku. 

\section{Motivace}
Motivace pro generativní programování vznikla současně se vznikem počítačů. Stroje nerozumí naší řeči a jejich funkce jsou pouhými elektrickými signály. Vznikla potřeba vymyslet způsob, kterým budeme procesoru předávat informace o tom, co se po něm vyžaduje. Psaní těchto instrukcí pomocí nul a jedniček je ale nepřehledné a pro člověka nesrozumitelné. Vznikly tedy první sady instrukcí procesorů, které byli pro člověka čitelné. Tyto instrukce byly překládány přímo do strojového kódu, podle kterého procesor pracoval. Tento překlad je právě generováním kódu. Instrukce jsou využity jako šablona a překladač generuje strojový kód na základě těchto instrukcí.

S rozmachem programování vznikl požadavek na vyšší abstrakci tak, aby se instrukce co nejvíc podobali lidské řeči. To vedlo ke vzniku velkého množství programovacích jazyků, překladačů a interpreterů, které tyto komplexnější instrukce překládají na strojový kód. Překladače jsou tedy generátory, které na základě šablony generují strojový kód. 

Generativní programování je o návrhu a implementaci softwarových modulů, které je možné zkombinovat a generovat tak specializované a rozsáhlé systémy splňující specifické požadavky [Cza98]. Cílem je zmenšit mezeru zdrojovým kódem a konceptem, dosáhnout vysoké znovupoužitelnosti a adaptability software a zjednodušit správu velkého množství variant komponenty a zvýšit efektivitu [CEG98].

\section{Generátory}
Podle [Cza98] jsou pro generování použity dvě základní metody - \textit{kompoziční} a \textit{transformační}. Generátory založené na kompozici se nazývají \textit{kompoziční generátory} a generátory založené na transformaci se nazývají \textit{transformační generátory}. Oba dva typy generátorů si ukážeme na příkladě, kde budeme vytvářet instanci hvězdy z různých komponent. Hvězda je sestavena z modelu vlastností, který obsahuje:
\begin{itemize}
	\item počet cípů
	\item vnitřní poloměr
	\item vnější poloměr
	\item úhel popisující naklonění prvního cípu
\end{itemize}
Nyní si ukážeme, jak k vytvoření hvězdy přistupují obě dvě metody.
\newline
\newline
ZDE BUDE OBRÁZEK CZARNECKI FIGURE 52
\newline
\newline

\subsection{Kompozice}
V kompozičním modelu spojíme několik komponent dohromady, které tak vytvoří požadovaný celek. K tomu, abychom byli schopni generovat různé hvězdy, je potřeba mít sadu konkrétních komponent různých velikostí a tvarů.
\newline
\newline
ZDE BUDE OBRÁZEK CZARNECKI FIGURE 55.
\newline
\newline
Kruh je popsán pouze vnitřním poloměrem, zatímco cípy jsou popsány vnitřním poloměrem, jejich počtem a vnějším poloměrem. K tomu abychom sestavili čtyřcípou hvězdu, která je vidět na obrázku, je potřeba jeden kruh a čtyři cípy vybrané ze sady konkrétních komponent, na základě jejich vlastností.
\newline
\newline
ZDE BUDE OBRÁZEK CZARNECKI FIGURE 53.
\newline
\newline
Efektivnějším způsobem sestavení instance je použití \textit{generativních komponent} místo konkrétních komponent. Generativní komponenta využívá abstraktního popisu komponent a generuje komponentu na základě popisu. Například místo celé sady všech možných kruhů a cípů potřebujeme dvě generativní komponenty, a to generativní kruh a generativní cíp. Generativní kruh má jako parametr vnitřní poloměr a generativní cíp má jako své parametry vnitřní poloměr, vnější poloměr a úhel.
\newline
\newline
ZDE BUDE OBRÁZEK CZARNECKI FIGURE 54
\newline
\newline
Konkrétní komponenty jsou následně vygenerovány a poskládány do požadovaného obrazce.

\subsection{Transformace}
Narozdíl od kompozice nespojujeme jednotlivé komponenty, ale provádíme určitý počet transformací, které vyústí v požadovaný výsledek. Není potřeba mít nadefinovanou sadu komponent, nebo jejich generování, ale je třeba mít nadefinované transformace, které můžeme s instancí provádět. V tomto příkladě jsou to transformace: 
\begin{itemize}
	\item přidej 4 cípy
	\item zvětš vnější poloměr
	\item otoč o 45 stupňů
\end{itemize}
ZDE BUDE OBRÁZEK CZARNECKI FIGURE 53

\subsection{Shrnutí}
I když se transformační generátor zdá být jednodušśí, většina generátorů, které známe, jsou kompoziční. Příkladem takového generátoru může být editor GUI, který na základě grafického editoru, ve kterém poskládáme grafické komponenty dohromady, vygeneruje kód, který je popisuje.

Kompoziční generátor bude použit v této práci. Na základě vytvořené varianty z modelu vlastností bude generován kód jazyka tesa. Komponenty zde budou zaznamenány jako záznamy v souboru formátu XML. Z těchto komponent bude následně vygenerován výsledný kód.


