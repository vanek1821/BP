\chapter{Modelovací nástroje}

V této kapitole se budeme zabývat existujícími nástroji, které umožňují modelování vlastností. Cílem je ukázat, jaké modelovací nástroje existují, k čemu se používají a zachytit rozsah jejich funkcí. Bude vysvětleno, jaké mají nástroje výhody a nevýhody a odůvodnění výběru nástroje.

\subsection{pure::variants}

Pure::variants je jeden z mála komerčně využívaných modelovacích nástrojů od společnosti pure-systems. Není zaměřen pouze na modelování vlastností, ale svou funkcionalitou se snaží pokrýt všechny fáze vývoje software. Samotný software je plug-inem do vývojového prostředí Eclipse. Umožňuje práci s několika modely a pro každý z těchto modelů má vlastní editor.

Hlavním modelem je \textit{Feature Model} nebo-li model vlastností. Software zobrazuje model vastností ve stromové architektuře. Umožňuje vytvoření čtyř různých závislostí: povinné, volitelné, alternativní a slučitelné. Umožňuje také přidávat závislosti vlastností napříč modelem. Například pokud bude ve finální variantě vybrána nějaká vlastnost, software automaticky vybere i další, která je na ní závislá a nedovolí vytvořit variantu těchto vlastností. 

Dalšími modely jsou \textit{Family model} nebo-li model rodiny produktů. Tento model elementy této rodiny a dokáže na ně namapovat vlastnosti. Dále \textit{Variant description model}, který vlastnosti dokáže konfigurovat. Posledním modelem je \textit{Variant result model}, který narozdíl od předchozích jako jediný nemá vlastní editor. Tento model popisuje konkrétní výstupní variantu produktu, její popis a informace pro její sestavení.

Pro naše účely budeme využívat \textit{Feature model}, který dokáže jednoduchým způsobem sestavit výsednou variantu z vytvořených vlastností pomocí zaškrtávání. 
\newline
\newline
ZDE BUDE OBRÁZEK PURE::VARIANTS
\newline
\newline

\subsubsection{Výhody}

Hlavní výhodou je uživatelsky přívětivé ovládání. Vytvoření modelu je intuitivní a jednoduché. Vlastnosti se přidávají pomocí kontextového menu a u každé nové vlastnosti je uživatel vyzván k popisu této vlastnosti. Z výsledného stromu se dá jednoduše vytvořit finální varianta a software si sám hlídá, zda jsou dodržena veškerá pravidla, která jsou v modelu zanesena.

Další velkou výhodou, která je pro tuto práci klíčová, je možnost importu a exportu modelu a výsledné varianty pomocí souborů ve formátech \textit{CSV}, \textit{XML}, \textit{HTML} a \textit{DOT}. 

\subsubsection{Nevýhody}

Hlavní nevýhodou tohoto software, je dle mého názoru vysoká cena za licenci. Oficiální cena není ani na oficiálních stránkách produktu uvedena a zákazník se jí dozví přímým e-mailovým dotazem na společnost. Naše společnost zakoupila několik licencí na tento software a cena jedné licence se pohybovala okolo 10 000 eur při čemž je potřeba zaplatit 20\% každý rok za prodloužení licence. 

Další nevýhodou může být přílišná komplexnost nástroje. Nástroj obsahuje opravdu velké množství funkcí a plně využít jeho potenciál může být náročné. 

\subsubsection{Shrnutí}
Nástroj Pure::variants je zástupcem komerčně využívaných nástrojů. Tento nástroj bude v práci využit k zachycení šablony díky jeho možnosti importu a exportu. Je pravděpodobobné, že pro účel ke kterému bude nástroj využíván by byl jiný nástroj vhodnější, ale jelikož firma ZF vlastní několik licencí a jedna z nich mi byla k účelům bakalářské práce poskytnuta, bude využit tento nástroj.

